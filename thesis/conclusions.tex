\chapter{Conclusions and perspectives}
\label{ch:conclusions}
In this chapter, we summarize how each of the research topics presented in the first chapter has been pursued, and the contributions which have resulted. Next, we reflect on how we can improve our contributions and provide new research directions.

\section{Conclusions}
At the beginning of this thesis we set out to identify which are the key components that would help us achieve true data ownership and interoperability for the next-gen social Web. While decentralisation is the most important factor of the equation, our model would not work unless true interoperability is achieved. For this reason, we decided to use Semantic Web technologies, as they help represent data in a way that cannot be confusing or misleading. Additionally, we participated in the World Wide Web Consortium, where we translated our research results into standards, allowing us to obtain immediate feedback from experts all around the world. Both WebID and WebID-TLS standards are currently under final review within the WebID group before being submitted for review to the W3C committee.\\

We have identified and pursued three important research topics, each dealing with an important aspect of a true decentralized social Web. In the order in which they were approached and researched, these topics are: identity, authentication and access control.

\subsection{Identity and authentication}
The first aspect relates to decentralized online identity, both in terms of identification as well as authentication. To this regard, we have contributed to WebID (Chapter~\ref{ch:identity}), a simple and universal identification mechanism that is distributed and openly extensible. It improves privacy, security and control over how each person can identify themselves on the Web. Additionally, when coupled with client certificates, WebID can be used for authentication purposes, as it is the case of WebID-TLS.\\

We have investigated existing protocols that claim to offer decentralized identity, such as OpenID Connect~\cite{sakimura2011openid}, Mozilla Persona~\cite{browserid} (formally known as BrowserID), Web Authentication~\cite{dahl2013web} and the SAML~\cite{hallam2001security} standard. Unfortunately, none of the aforementioned protocols offer true decentralization, since service providers offer a limited number of choices for identity providers, based on a priori trust relationships between them. Additionally, even though some user attributes are transmitted once authentication has been performed, the user is usually forced to create a local account on the new service provider, thus moving from one silo to another. Moreover, username and passwords were still the preferred choice when it comes to authentication credentials, effectively decreasing the security of the system.\\

\subsubsection{WebID}
The general idea behind WebID is that Agents (e.g. a person, an organization, a group, etc.) create their own identities by linking a \textit{unique identifier} (i.e. an HTTP URI) to a \textit{profile document}, using a standardized RDF serialization format in order to provide interoperability. From the WebID owner's point of view, hosting the profile document on a device directly controlled by the user increases privacy and data ownership. However, several issues now arise, dealing primarily with privacy and trust.\\

By aggregating user data from multiple services, attackers could in theory be able to fingerprint and build a complete profile of the user. Additionally, relying on HTTPS to trust the data origin also implies having to validate the server's certificates through a standard PKI trust chain, which is fundamentally vulnerable because CAs can be compromised and used to replace valid certificates for millions of domains. We have presented several solutions to these problems in Section~\ref{subsec:webid_limitations} of Chapter~\ref{ch:identity}, which rely mainly on using HTTPS.\\

\subsubsection{WebID-TLS}
The WebID-TLS authentication protocol enables secure, efficient and user friendly authentication on the Web. Instead of using passwords, it allows people to choose a client certificate proposed to them by their browser during the authentication process. Even thought it is based on TLS, it does not rely on Certification Authorities. To build trust, it uses the Semantic Web to reason over the contents of the profile document, where a web of trust can be created from the relations a user has with different people. Basically, to trust someone, it may be sufficient that one or more of my friends trust the same person, by expressing this relation in the form of a \textit{FOAF:knows} relation. It is up to each service provider to decide on the granularity on which it grants access to users.\\

WebID-TLS relies on client certificates to prove that an agent possesses the private key that matches a public key stored in the WebID profile document. This also implies that only the owner of the private key has write access to the profile document and thus it is capable of adding an RDF description of his/her public key.\\

As was the case for WebID, a WebID-TLS verifier must trust the source of the profile document and also that no man-in-the-middle attack is currently taking place. Furthermore, the verifier must also trust the authenticity of a profile document's contents. Even though HTTPS can be used to ensure end-to-end data security, an attacker might have compromised the origin server through other means, and therefore was able to alter the contents of the profile document in order to insert his/her own public key. A solution would be to sign the contents of the profile document through a cryptographic mechanism as the one provided by the GNU Privacy Guard (GPG)~\cite{koch2003gnu}. However, this approach not only implies an existing Web of Trust between the parties, but also the ability to perform cryptographic operations on the triples. Unfortunately, up to now we are unable to provide an abstract/canonical representation of triples, independent of existing serialization schemes. Therefore, signing the contents of a profile document may provide unpredictable results and is not advisable.

\subsection{Access control}
The basic concepts upon which an access control model is based determines the flexibility of the model to adapt to different environments and systems. We are interested in applying access control to decentralized systems, where interoperability and data portability are the decisive factors. To this regard, we have proposed an access control model for the social Semantic Web, which takes into account the dynamic evolution of user relations, and which applies to Linked Data generated by users (e.g. profile data, wall posts, conversations, etc.).\\

While compiling a list of related work, we realised that most of the standard access control models were developed for closed or centralized (i.e. \textit{silo}) environments, where the same entity is responsible for the assignment of attributes or privileges to clients and the evaluation of the access requests to determine whether they must be granted or not. Furthermore, they did not offer an easy way for exporting access control policies along with the user's data. We were only able to find four existing solutions that would match our criteria. The first one is Web Access Control (WAC)~\cite{hollenbach2009using}, a decentralized system in which different users and groups are given various forms of access to resources, and where users and groups are identified by HTTP URIs. Next, Accountability in RDF (AIR)~\cite{kagal2011gasping} is a Semantic Web-based rule language that provides access control while focusing on generating explanations for its inferences and actions as well as conforming to Linked Data principles. Social Semantic Web Access Control~\cite{villata2011social} is based on the \textit{Social Semantic SPARQL Security for Access Control} vocabulary, allowing users to specify fine-grained access control policies for their RDF data. Finally, Shi3ld offers a context-aware access control framework for consuming the Web of Data from mobile devices.\\

We believe that current access control mechanisms suffer from a lack of \textit{context}, specific to relationships between people, since defining a set of static policies may not equally apply in every situation. Additionally, static policies do not take into consideration the dynamics of human relationships and the speed at which they evolve. Furthermore, most semantic access control systems only apply to resources in the form of documents.\\

We consider that depending on different contexts and situations, access to resources can be handled differently. For this reason, we have proposed a social distance metric called \textit{proximity}, which also plays a very important role in our model. We have used both context and proximity as metrics to propose a dynamic access control system tailored for social Web applications. In order to describe relationships between users, we were inspired by T. Hall's book, The Hidden Dimension\cite{edward1966hall}, to define four social proximity levels, namely \verb+Public+, \verb+Social+, \verb+Close+ (corresponding to the Personal level), and \verb+Intimate+. Since these proximity levels do not provide context by themselves, we decided to utilize the concept of labels (e.g. \verb+#family+, \verb+#sportsteam+) that can be then assigned to each proximity level.\\

Our proposed solution, the \textit{Social Access Control Service} (SACS) is orthogonal to existing access control systems, as it uses proximity levels and contexts together with our online interactions (e.g. sharing a resource with someone, tagging a user within a specific context, etc.), in order to apply access control policies to a user's online social resources (e.g. a user's profile, wall posts, conversations).

\subsubsection{Social Access Control Service}
The Social Access Control Service is comprised of two distinct sub-services, a \textit{Relationship Monitor} (RM) engine and a \textit{Static Access Control} (SAC) engine, each with a particular set of tasks (Section~\ref{sec:sacs} of Chapter~\ref{ch:ac}).\\

The RM engine relies on metrics such as \textit{context} and \textit{proximity distance} together with a relationship history between the involved actors, in order to either provide notifications for potential privacy issues that may arise when disclosing information, or even to modify existing access control policies for incoming requests (if so configured). The RM applies to two distinct types of actions. First, it is used when assigning context labels when disclosing information, and second for handling a request for a resource.\\

The SAC engine handles access to user-generated resources, based on predefined privacy policies. The process of permitting/denying access to a resource is quite straightforward, and it involves matching the user performing the request to a list of resources matching the same \textit{context} label. The goal of this process is to finally return a \textit{unique view} of requested resources (e.g. a user's profile), matching to the level of access corresponding to the requesting user.\\

We regret not having sufficient time to propose additional metrics, as they would improve the granularity of access control rules. Furthermore, as we were unable to implement the RM engine in a real application environment, we have no estimate of how well it scales, nor how user friendly it would be in terms of policy management.


\subsection{Validating the proposed solutions}
As part of our efforts to validate solutions and standards to which we have contributed, we have attempted to implement several Web services that would incorporate decentralized user identity, secure authentication, semantic data storage. To respect the initial requirements, they must be able to apply Create-Read-Update-Delete (CRUD) operations to resources, to offer increased privacy through access control, and most importantly, to be interoperable with other applications in terms of data exchange (e.g. content sharing, messaging, activity notifications, etc.).\\

Following the order in which the research was conducted, we began with MyProfile\footnote{http://myprofile-project.org/}, a service that offers a unified user account, centralizes the user's data and puts it under the user's control, and more importantly on a device the user controls. MyProfile offers not only the possibility to create and manage WebIDs, but also to authenticate users through WebID-TLS in order to gain access to personalized services like \textit{personal walls}, as well as \textit{messages} and \textit{notifications} between users.\\

According to the two different WebID-TLS authentication approaches, either performing the WebID-TLS verification locally, or using a third party WebID-TLS authentication service, we have developed two libraries, written in PHP, which cover both approaches and can be used together. 

Next, we have implemented a Static Access Control engine, which was released as a public library together with a caching module. The SAC engine is used to offer a \textit{unique view} of a user's profile, based on the level of access specific for the agent requesting the information. It uses context labels to match users to the resources to which they are allowed access.\\

Finally, we have implemented a Linked Data personal data store platform called RWW.I/O\footnote{https://rww.io/}, which operates under the assumption that users require a personal data store, where different applications can store data about and for the user, and where data are equally available between applications. The platform supports full Create-Read-Update-Delete (CRUD) operations, complying to REST standards. Documents and directories can be created by performing HTTP requests such as POST, PUT and MKCOL (i.e. new directories), following the requirements we presented at the beginning of this thesis. This work was conducted at the Massachusetts Institute of Technology and it was supervised by Sir Tim Berners-Lee.\\

The source code for all the services and libraries we have implemented is publicly available on GitHub\footnote{http://github.com/}, under an MIT license.

\section{Perspective work}
\label{sec:perspectives}

During the research and implementation stages of this thesis, we were able to identify several research directions that we were unable to pursue at that moment. Among them, the most important ones concern how applications can notify users when events occur as well as allowing users to safely communicate with one another while maintaining an adequate level of privacy. Furthermore, by using persistent identifiers, we became worried that applications and services will be able to track users, and therefore we began to envision the possibility of creating an identity proxy service. Next, we would like to shortly present a preliminary description of these perspective research directions.

\subsection{Semantic Messaging and Notifications Protocol}
As the name suggests, the Semantic Messaging and Notifications Protocol (SMNP) intends to offer two distinct types of messages. The first type is similar to email, in that it provides end-to-end communication between the involved parties. The second type consists of activity streams (e.g. updates about other users).\\

In both cases, a special resource is created, offering a semantic representation of the message. The resource would normally contain a sender, one or more recipients, and a message body.

\subsubsection{Semantic messaging}
\label{subsubsec:messaging}
Semantic messaging provides the means to enable written communication between two parties. It intends to offer an alternative to email, operating in two modes, pull and push, depending on the user's preferences.\\

Users may also choose which kind of mode they allow when receiving messages. For instance, users may allow their friends to send messages using the push method, while disallowing incoming messages from people they know. On the other hand, messages sent using the pull mode may always be accepted.\\

The \textbf{pull mode} involves creating messages, storing them locally and then notifying the other endpoints with the URI of the resource that contains the message. This mode requires a dedicated service, capable of centralizing the messages in one point, on a device controlled by the sender.\\

The main advantage offered by this mode concerns the privacy of message data, since messages are centralized on the sender's server, and only a link to the message is sent to the recipients, therefore the sender remaining in control of his/her data. Compared to classic email, users will still be able to "save" (create a local copy of the resource) when they access the message, though they would have to successfully authenticate before being allowed to access the message contents. The drawback here is that if the server on which the messages are stored is no longer accessible, no one will be able to read the messages.\\

Another advantage of self-hosted messages is that it eliminates the extra network traffic generated by spam, since senders will have to take care of the scalability issues involved when hosting the message contents. Not sending the message contents also reduces the risks of getting infected with viruses when opening the message, as users would now have to explicitly click on the link in order to fetch and read the message.\\

The \textbf{push mode} implies sending messages to remote endpoints. This decentralized operation mode is very similar to email, as it transmits messages to one or more remote recipients, and is suitable for users who do not possess a server that is always available online. However, recipients must be able to receive messages, meaning that they must permit access to a messaging endpoint, typically a resource with "Write" or "Append".\\

The advantage of this mode is that once the messages have been delivered, the sender no longer needs to worry whether his/her server will be available or not. Since this mode involves duplicating data when transmitting the message to each recipient (copying the same message for each recipient), the user must wait for the sending process to complete, which may take a long period of time due as it requires more bandwidth (especially when sending large documents).\\

To offer the same functionality as the one provided by email when sending attachments, SMNP would involve performing one ore more POST requests containing the resource to the recipients' servers, where a service capable of interpreting and managing these requests must be present.

\subsubsection{Notifications}
\label{subsubsec:notifications}
Notifications are short messages, sometimes as short as a several triples, stating that a resource is available at a given location (URI), or that there is an event concerning a specific user (e.g. "Barry is now friends with Ann"). Notifications can also be split based on two operation modes, pull and push.\\

In \textbf{pull mode}, the notification sender is only required to POST a single triple of type \textit{event} with a URI pointing to the remote event resource, indicating that more information about the event can be found at a given location. If the recipient of the notification message would like to know more about this event, he/she can fetch more information from that location.\\

In \textbf{push mode}, the notification sender may also provide more information about the event at the time of transmission, in the form of a description, a date, the WebIDs of the concerned parties, etc..

\subsection{Transparent WebID proxy}
\label{subsec:webid_proxy}
Avoiding traceability and fingerprinting are important factors, directly impacting the level of privacy offered by a WebID provider (Privacy and security analysis in Chapter~\ref{ch:identity}). Even though access control can be successfully utilized to restrict access to parts of a profile document (Chapter~\ref{ch:ac}), given a reasonable amount of time, a service provider will be able to build a complete profile of a user. To avoid being traced and fingerprinted across different applications and services, a common solution would be to use multiple identities. However, this solution increases the difficulty of identity management for most users, while others may even feel inclined to trade their privacy for spending less time having to manage their alias identities.\\

We believe we can offer an alternative, in the form of a transparent WebID proxy, a proxy which intercepts normal communication without requiring any special client configuration, and where clients need not be aware of its existence. Please consider the following example.\\

Barry has his own WebID \verb+https://barry.example/profile#me+, which is hosted at https://barry.example. Instead of using his own WebID, he could use a transparent WebID proxy service located at https://webid.proxy. The proxy service would then allow Barry to create a new identity, a so called pseudo-WebID (i.e. \verb+https://webid.proxy/my-alias/card#me+), which would only contain the profile elements Barry has selected from his real WebID. When a client dereferences the pseudo-WebID, the proxy would fetch the elements from Barry's real profile and serve the profile document on the fly. If at any point Barry decides to stop using the service, he can then create an ACL rule to block https://webid.proxy from accessing his WebID.\\

Please note that using such a service implies trusting the service provider to not disclose the link between the pseudo-WebID and the real WebID. As a security measure, users could allow access to their real WebID profiles only to a selected number of agents by default, among which we can find the proxy service. This way, even if an attacker discovers the user's real WebID, he will not be able to gain access to the user's real profile document, as long as the user has removed the proxy service from the list of authorized agents.







